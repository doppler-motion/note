\documentclass{paper}
\usepackage{color}
\usepackage{ctex}
\usepackage{geometry}

% 设置页面
\geometry{a4paper, left=2cm, right=2cm, top=2cm, bottom=2cm}


\begin{document}
	\part{Python}
	\section{基础}
	\subsection{列表和数组的区别}
	\begin{enumerate}
		\item 列表
		\begin{enumerate}
			\item[a] 元素按特定顺序排列组成
			\item[b] 元素可以是多种类型;
			\item[c] 可以方便、高效的添加删除元素;
			\item[d] 不可以进行数组四则运算;
			\item[e] 相比数组有更多的存储空间;
			\item[f] 列表中是数据存放的地址,即指针
		\end{enumerate}
		\item 数组
		\begin{enumerate}
			\item[a] 数据类型必须一样;
			\item[b] 可以进行数学四则运算;
			\item[c] 具有广播功能;
			\item[d] 执行速度快;
			\item[e] 节省空间
		\end{enumerate}
	\end{enumerate}
	
	\section{web框架}
	
	\part{算法}
	\section{基础算法知识}
	\begin{enumerate}
		\item 过拟合是什么、原因、解决方法
		\begin{enumerate}
			\item 表现:模型的泛化能力差,在训练数据表现好,验证数据及其他数据表现差
			\item 原因
			\begin{enumerate}
				\item 训练数据量不足,样本类型单一
				\item 训练集存在噪声
				\item 模型复杂度过高
			\end{enumerate}
			\item 解决方案
			\begin{enumerate}
				\item 使样本尽可能的均衡
				\item 降低模型训练复杂度
				\item 正则化
				\item 交叉检验
				\item dropout
			\end{enumerate} 
		\end{enumerate}
		
		\item 欠拟合是什么、原因、解决方法
		\begin{itemize}
			\item 
		\end{itemize}
		
	\end{enumerate}
	\section{机器学习}
	\section{深度学习}
	\section{NLP算法}
	
	\part{数学}
	\section{概率论与统计}
	\subsection{方差、偏差是什么?}
	\begin{enumerate}
		\item 方差
		\begin{enumerate}
			\item 反映预测值之间的关系,即它们之间的离散程度
		\end{enumerate} 
		\item 偏差
		\begin{enumerate}
			\item 反映预测值和真实值之间的差值
		\end{enumerate}
	\end{enumerate}
	
	\part{计算机基础}
	\section{计算机网络}
	\section{数据库}
	\section{docker}
\end{document}